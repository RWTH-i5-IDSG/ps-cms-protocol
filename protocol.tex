% %
% LAYOUT_E.TEX - Short description of REFMAN.CLS
%                                       99-03-20
% http://www.ctan.org/pkg/refman
%
% Updated for REFMAN.CLS (LaTeX2e)
%
\documentclass[a4paper]{refart}
\usepackage[utf8]{inputenc}
\usepackage[T1]{fontenc}
\usepackage[english]{babel}
\usepackage{lmodern}

\usepackage{makeidx}
\usepackage{acronym}
\usepackage[colorlinks=false]{hyperref}
\usepackage{tabularx}
\usepackage{colortbl}
\usepackage[table]{xcolor}

% New page after each section
\usepackage{titlesec}
\newcommand{\sectionbreak}{\clearpage}

% This will reset the horizontal layout, giving you 
% a textwidth of fraction*fullwidth.
\settextfraction{0.75}

% more at http://latexcolor.com/
\definecolor{table-head}{rgb}{0.98, 0.92, 0.84}

\usepackage[pdftex]{graphicx}
\graphicspath{{./figures/}}

\title{Communication Protocol Between Pedelec Station and Central Management System}
\author{
Site:\\
\url{github.com/RWTH-i5-IDSG/ps-cms-protocol}\\ \vspace{5mm}
Authors:\\
RWTH - i5\\ \vspace{5mm}
Version:\\
0.0.21\\ \vspace{5mm}
Date:\\
28.07.2015}

\date{}
\emergencystretch1em  %

\pagestyle{myfootings}

\setcounter{tocdepth}{2}
\renewcommand{\arraystretch}{1.3}

\makeindex 

\begin{document}
\maketitle

\section*{Version History}

\begin{tabularx}{\linewidth}{ | l | l | X | }
  \hline
  \rowcolor{table-head}
  Version & Date & Description \\
  \hline
  0.0.1	& \date{01.07.2014} & Initial draft \\
  0.0.2	& \date{03.07.2014} & Added Update Firmware (by \acs{CMS}) and Firmware Status Notification (by \acs{PS}) \\
  		& 				   & Added Change \& Get Pedelec Configuration (by \acs{CMS}) \\
  0.0.3 & \date{08.07.2014} & Deleted maxBatteryRange from Charging Status Notification (by \acs{PS}) \\
  0.0.4 & \date{10.07.2014} & Added Upload Logs (by \acs{CMS}) and Logs Status Notification (by \acs{PS}) \\
  		& 				   & Added more parameters to Charging Status Notification (by \acs{CMS}) \\
  0.0.5 & \date{18.08.2014} & Updated station pedelec URLs in \autoref{cms:main} \\
  0.0.6 & \date{25.08.2014} & Added ChangeState Type \\
  0.0.7 & \date{06.10.2014} & Added the operations Reserve Now and Cancel Reservation\\
  0.0.8 & \date{15.10.2014} & Changed in "Start Transaction" parameter from 'userId' to 'cardId'\\
  0.0.9 & \date{24.10.2014} & Send 'cardId' instead of 'userId' after Authorize Request\\
  0.0.10 & \date{03.11.2014} & Add Card Activation API\\
  0.0.11 & \date{27.11.2014} & Add "Get Available Pedelecs" API and update "Card Activation" API\\
  0.0.12 & \date{12.01.2015} & Change request param in "Remote Authorize"; replacing "remainingTrials" in "Authorize" with error 403\\
%  0.0.13 & \date{03.03.2015} & Add "callbackUrl" param to BootNotification\\
  0.0.14 & \date{04.03.2015} & In 2.7 \& 2.8 renamed to "cardPin", added types-column, changed "userId" to "cardId" in 3.14; "pedelecManufacturerId" in 2.1 and "errorcode"/"errorinfo" in 2.2/2.3 are now optional; removed "slotManufacturerId" in 2.4\\
  0.0.15 & \date{06.03.2015} & Added EncryptionKeys in 2.1 BootNotification and checksum in 3.12 Update Firmware\\
  0.0.16 & \date{10.03.2015} & Renamed values in chapter 4 - Types and errorCode + errorInfo are optional \\
  0.0.17 & \date{24.04.2015} & Added stationURL in BootNotification and accountState in Authorize (Chapter 2)\\
  0.0.18 & \date{24.06.2015} & Simplified return object of get available pedelecs\\
  \hline
\end{tabularx}

\tableofcontents

\section*{Acronyms}

\begin{acronym}[Bash]
 \acro{PS}{Pedelec Station}
 \acro{CMS}{Central Management System}
\end{acronym}
\section{Introduction}

// TODO

\subsection{Technology}

The protocol is designed to be implemented as a RESTful Webservice with HTTP as the underlying data transfer protocol. The resources are represented in JSON data format.
\section{Operations Initiated by Pedelec Station}

\subsection{Boot Notification}

\marginlabel{Description:}
After start-up of a \acs{PS}, the \acs{PS} sends a notification to the \acs{CMS} with information about its configuration (e.g., manufacturer id, connected station slots and pedelecs). \acs{CMS} will accept only registered stations. 

After each reboot, the Boot Notification is sent.

The \acs{CMS} sends a response with the acceptable status, including current time and heartbeat interval if accepted.

The \acs{PS} repeats the Boot Notification (in an appropriated interval) until the \acs{CMS} accepts the \acs{PS}. The \acs{PS} requests nothing else, until \acs{CMS} accepts it.


\marginlabel{URL:} \url{[BASE_CMS_URI]/boot}

\marginlabel{Method:} POST

\marginlabel{Request:}
\begin{table}[!h]
\vspace{-7mm}
\begin{tabularx}{\linewidth}{ | l | l | X | }
  \hline
  \rowcolor{table-head}
  Field & Type & Description \\
  \hline
  stationManufacturerId & String 		& This value identifies the \acs{PS} by its hardware serial \\
  firmwareVersion			& String & Firmware version of \acs{PS}\\
  slots/slotManufacturerId 	& String & This value identifies the slot by its hardware serial \\
  slots/slotPosition			& Integer & The sequence position of the connected slot \\
  slots/pedelecManufacturerId & String (optional) & This value identifies the locked pedelec by its hardware serial related to the slot \\
  %callbackUrl				& String & URL to communicate with \acs{PS} initiated by \acs{CMS}\\
    \hline
\end{tabularx}
\end{table}

\marginlabel{Response:}
\textbf{200 OK}

\begin{tabularx}{\linewidth}{ | l | l | X | }
  \hline
  \rowcolor{table-head}
  Field & Type & Description \\
  \hline
  timestamp & Long & Unix timestamp (seconds since epoch) \\
  heartbeatInterval & Integer & In seconds \\
    \hline
\end{tabularx}

\marginlabel{Errors:} 
\textbf{406 Not Acceptable} (If the station is not registered)

\textbf{5xx Server Error}

\subsection{Station Status Notification}

\marginlabel{Description:}
A \acs{PS} sends a notification to the \acs{CMS} to inform the \acs{CMS} about its status or error condition within the \acs{PS} including the connected station slots. A \acs{PS} shall send a Station Status Notification when it becomes unavailable as a result of an error condition or other external events.

\marginlabel{URL:} \url{[BASE_CMS_URI]/status/station}

\marginlabel{Method:} POST

\newpage
\marginlabel{Request:}
\begin{table}[!h]
\vspace{-7mm}
\begin{tabularx}{\linewidth}{ | l  | l | X | }
  \hline
  \rowcolor{table-head}
  Field & Type & Description \\
  \hline
  stationManufacturerId & String 		& This value identifies the \acs{PS} by its hardware serial\\
  stationErrorCode & String & \\
  stationErrorInfo & String & \\
  stationState & String & See Section \ref{types:OperationState} \\
  timestamp & Long & Unix timestamp (seconds since epoch) \\
  slots/slotManufacturerId & String 	& This value identifies the slot by its hardware serial \\
  slots/slotErrorCode & String & Optional when OPERATIONAL\\
  slots/slotErrorInfo & String & Optional when OPERATIONAL\\
  slots/slotState & String & See Section \ref{types:OperationState} \\
  \hline
\end{tabularx}
\end{table}

\marginlabel{Response:} \textbf{200 OK}

\marginlabel{Errors:} \textbf{5xx Server Error}

\subsection{Pedelec Status Notification}

\marginlabel{Description:}
A \acs{PS} sends a notification to the \acs{CMS} to inform the \acs{CMS} about the status or error condition of connected pedelecs. A \acs{PS} shall send an Pedelec Status Notification when a pedelec becomes unavailable as a result of an error condition or other external events.

\marginlabel{URL:} \url{[BASE_CMS_URI]/status/pedelec}

\marginlabel{Method:} POST

\marginlabel{Request:}
\begin{table}[!h]
\vspace{-7mm}
\begin{tabularx}{\linewidth}{ | l | l | X | }
  \hline
  \rowcolor{table-head}
  Field & Type & Description \\
  \hline
  pedelecManufacturerId & String 		& This value identifies the Pedelec by its hardware serial\\
  pedelecErrorCode & String & Optional when OPERATIONAL \\
  pedelecErrorInfo & String & Optional when OPERATIONAL\\
  pedelecState & String & See Section \ref{types:OperationState} \\
  timestamp & Long & Unix timestamp (seconds since epoch) \\
  \hline
\end{tabularx}
\end{table}

\marginlabel{Response:} \textbf{200 OK}

\marginlabel{Errors:} \textbf{5xx Server Error}


\subsection{Charging Status Notification}
\label{ps:charging-status}

\marginlabel{Description:}
The \acs{PS} informs the \acs{CMS} at regular intervals about the charging status (in time intervals or when fully charged) of all connected pedelecs of the \acs{PS}.

The message contains the current timestamp, the meter value (Wh), the charging state (e.g., charging, completed), the and PedelecID.

\marginlabel{URL:} \url{[BASE_CMS_URI]/status/charging}

\marginlabel{Method:} POST

\newpage
\marginlabel{Request:}
\begin{table}[!h]
\vspace{-7mm}
\begin{tabularx}{\linewidth}{ | l | l | X | }
  \hline
  \rowcolor{table-head}
  Field & Type & Description \\
  \hline
  	pedelecManufacturerId & String			& \\
  	timestamp & Long			& Unix timestamp (seconds since epoch) \\
  	chargingState & String		& See Section \ref{types:ChargingState} \\	
  	meterValue & Double & \\
  	battery/soc & Double & percentage points \\
  	battery/temperature & Double &  \\
  	battery/cycleCount & Integer &  \\
  	battery/voltage & Double &  \\
  	battery/current & Double &  \\
    \hline
\end{tabularx}
\end{table}

\marginlabel{Response:} \textbf{200 OK}

\marginlabel{Errors:} \textbf{5xx Server Error}

\subsection{Firmware Status Notification}

\marginlabel{Description:}
A \acs{PS} notifies \acs{CMS} about the success/failure of the firmware update. 

\marginlabel{URL:} \url{[BASE_CMS_URI]/status/firmware}

\marginlabel{Method:} POST

\marginlabel{Request:} 
\begin{tabularx}{\linewidth}{ | l | l | X | }
  \hline
  \rowcolor{table-head}
  Field & Type & Description \\
  \hline
  	status & String			& Progress status of the firmware update; see Section \ref{types:FirmwareState} \\	
  \hline
\end{tabularx}

\marginlabel{Response:} \textbf{200 OK}

\marginlabel{Errors:} \textbf{5xx Server Error}

\subsection{Logs Status Notification}

\marginlabel{Description:}
The \acs{PS} informs the \acs{CMS} about the status of requested uploading of logs.

\marginlabel{URL:} \url{[BASE_CMS_URI]/status/logs}

\marginlabel{Method:} POST

\marginlabel{Request:} 
\begin{tabularx}{\linewidth}{ | l | l | X | }
  \hline
  \rowcolor{table-head}
  Field & Type & Description \\
  \hline
  	status & String			& Upload status of the logs; see Section \ref{types:LogUpdateState} \\	
  \hline
\end{tabularx}

\marginlabel{Response:} \textbf{200 OK}

\marginlabel{Errors:} \textbf{5xx Server Error}

\subsection{Authorize}

\marginlabel{Description:}
Before a user can choose and unlock a pedelec with his CustomerCard (e.g., Bluecard), the \acs{PS} needs to be able to authorize the operation. Only after authorization the \acs{PS} will be able to unlock the pedelec. For this purpose the \acs{PS} needs user's Card-ID and PIN for authorization.

The response shall indicate, whether or not the Card-ID and PIN combination is accepted by the \acs{CMS}.

\marginlabel{URL:} \url{[BASE_CMS_URI]/authorize}

\marginlabel{Method:} POST

\marginlabel{Request:}
\begin{table}[!h]
\vspace{-7mm}
\begin{tabularx}{\linewidth}{ | l | l |  X | }
  \hline
  \rowcolor{table-head}
  Field & Type & Description \\
  \hline
  	cardId & String 		& Card specific number\\
  	cardPin & String			& User's secret PIN \\
    \hline
\end{tabularx}
\end{table}

\marginlabel{Response:}
\textbf{202 Accepted} (If credentials are accepted)

\begin{table}[!h]
\begin{tabularx}{\linewidth}{ | l | l | X | }
  \hline
  \rowcolor{table-head}
  Field & Type & Description \\
  \hline
  	cardId & String 		& Card specific number\\
%  	trialsRemaining & Count of remaining authentication trials, before the account gets disabled.\\
    \hline
\end{tabularx}
\end{table}

\marginlabel{Errors:} \textbf{403 Forbidden - no trials remaining and account gets disabled}

\textbf{406 Not Acceptable}

\textbf{5xx Server Error}

\subsection{Activate Card}

\marginlabel{Description:}
Before a user can use his card (e.g., Bluecard) to rent a bike, he has to activate it on the \acs{PS} terminal. For this purpose, the \acs{PS} sends Activation-Key and PIN to the \acs{CMS}.

The response shall indicate, whether or not the Activation-Key and PIN are accepted by the \acs{CMS} and responses with the CardId.

\marginlabel{URL:} \url{[BASE_CMS_URI]/activate-card}

\marginlabel{Method:} POST

\marginlabel{Request:}
\begin{table}[!h]
\vspace{-7mm}
\begin{tabularx}{\linewidth}{ | l | l | X | }
  \hline
  \rowcolor{table-head}
  Field & Type & Description \\
  \hline
  	activationKey & String & Key to start activation process to initial user's card\\
  	cardPin & String & PIN for customer's card\\
    \hline
\end{tabularx}
\end{table}

\marginlabel{Response:}
\textbf{202 Accepted} (If credentials are accepted)

\begin{table}[!h]
\begin{tabularx}{\linewidth}{ | l | l | X | }
  \hline
  \rowcolor{table-head}
  Field & Type & Description \\
  \hline
  	cardId 		& String & Card specific number\\
    \hline
\end{tabularx}
\end{table}

\marginlabel{Errors:} \textbf{406 Not Acceptable}

\textbf{5xx Server Error}

%\subsection{Set Card-Pin}
%
%\marginlabel{Description:}
%
%After a successful validation of the activation-key, the user can type in a pin for his card. At the end of the process, the \acs{PS} initializes the card with the userId. Is the copy procedure finished, the \acs{PS} sends cardId and cardPin to the \acs{CMS}.
%
%The response shall indicate, whether or not the card pin is successfully updated.
%
%\marginlabel{URL:} \url{[BASE_CMS_URI]/set-card-pin}
%
%\marginlabel{Method:} POST
%
%\marginlabel{Request:}
%\begin{table}[!h]
%\vspace{-7mm}
%\begin{tabularx}{\linewidth}{ | l | X | }
%  \hline
%  \rowcolor{table-head}
%  Field & Description \\
%  \hline
%  	cardId & Card Id to identifies user's card\\
%  	cardPin & Card Pin to set pin for user's card\\
%    \hline
%\end{tabularx}
%\end{table}
%
%\marginlabel{Response:}
%\textbf{200 OK} (If credentials are accepted)
%
%\marginlabel{Errors:} \textbf{406 Not Acceptable}
%
%\textbf{5xx Server Error}


\subsection{Start Transaction}

\marginlabel{Description:}
When the rental is authenticated, the station slot is unlocked and the user took the pedelec out of the slot, the \acs{PS} needs to inform the \acs{CMS} about this.

As response the \acs{CMS} sends an acknowledgment.

\marginlabel{URL:} \url{[BASE_CMS_URI]/transaction/start}

\marginlabel{Method:} POST

\marginlabel{Request:} 
\begin{table}[!h]
\vspace{-7mm}
\begin{tabularx}{\linewidth}{ | l | l | X | }
  \hline
  \rowcolor{table-head}
  Field & Type & Description \\
  \hline
  	cardId & String 		& \\
  	pedelecManufacturerId & String			& \\
  	stationManufacturerId & String			& \\
  	slotManufacturerId & String			& \\
  	timestamp & Long					& Unix timestamp (seconds since epoch) \\
    \hline
\end{tabularx}
\end{table}

\marginlabel{Response:} \textbf{200 OK}

\marginlabel{Errors:} \textbf{5xx Server Error}

\subsection{Stop Transaction}

\marginlabel{Description:}
After the \acs{PS} recognizes the return of a pedelec at a station slot, it needs to inform the \acs{CMS} about this.

As response the \acs{CMS} sends an acknowledgment.

\marginlabel{URL:} \url{[BASE_CMS_URI]/transaction/stop}

\marginlabel{Method:} POST

\marginlabel{Request:} 
\begin{table}[!h]
\vspace{-7mm}
\begin{tabularx}{\linewidth}{ | l | l | X | }
  \hline
  \rowcolor{table-head}
  Field & Type & Description \\
  \hline
  	pedelecManufacturerId			& String & \\
  	stationManufacturerId			& String & \\
  	slotManufacturerId			& String & \\
  	timestamp					& Long & Unix timestamp (seconds since epoch) \\
  	
    \hline
\end{tabularx}
\end{table}

\marginlabel{Response:} \textbf{200 OK}

\marginlabel{Errors:} \textbf{5xx Server Error}

\subsection{Get Available Pedelecs}

\marginlabel{Description:}
\acs{PS} can retrieve a list of available pedelecs ordered by a predefined priority.

\marginlabel{URL:} \url{[BASE_CMS_URI]/available-pedelecs}

\marginlabel{Method:} GET

\marginlabel{Request:} -

\marginlabel{Response:} \textbf{200 OK}

\begin{tabularx}{\linewidth}{ | l | l | X | }
  \hline
  \rowcolor{table-head}
  Field & Type & Description \\
  \hline
  	pedelecManufacturerIds			& String[] & List of manufacturer Ids for available pedelecs \\	
  \hline
\end{tabularx}

\marginlabel{Errors:} \textbf{5xx Server Error}

\subsection{Heartbeat}

\marginlabel{Description:}
To let the \acs{CMS} know that a station is still connected, a \acs{PS} sends heartbeats regularly in configurable time intervals.

The \acs{CMS} sends a response with the current time of the \acs{CMS}, which could be used to synchronize the time of the \acs{PS} with the time of the \acs{CMS}.

\marginlabel{URL:} \url{[BASE_CMS_URI]/heartbeat}

\marginlabel{Method:} GET

\marginlabel{Request:} -

\marginlabel{Response:} \textbf{200 OK}

\begin{tabularx}{\linewidth}{ | l | l | X | }
  \hline
  \rowcolor{table-head}
  Field & Type & Description \\
  \hline
  	timestamp & Long			& Unix timestamp (seconds since epoch) \\	
  \hline
\end{tabularx}


\marginlabel{Errors:} \textbf{5xx Server Error}
\section{Operations Initiated by Central Management System}

\subsection{Change Station Operation State}

\marginlabel{Description:}
\acs{CMS} can request to change the operation state of a \acs{PS} or its slots. The \acs{PS} can accept or reject the process the request. When rejected, the \acs{PS} must include a reason.

\marginlabel{URL:} \url{[BASE_PS_URI]/operation-state/station}

\marginlabel{Method:} PUT

\marginlabel{Request:} 
\begin{table}[!h]
\vspace{-7mm}
\begin{tabularx}{\linewidth}{ | l | X | }
  \hline
  \textit{Field} & \textit{Description} \\
  \hline \hline
  slotPosition (optional) 		& When present, the state of the slot with the given position will be changed. When absent, the state of whole \acs{PS} will be changed. \\
  state 					& See Section \ref{types:OperationState} \\
    \hline
\end{tabularx}
\end{table}

\marginlabel{Response:} \textbf{202 Accepted}

\marginlabel{Errors:} \textbf{406 Not Acceptable}

\textbf{5xx Server Error}

\subsection{Change Pedelec Operation State}

\marginlabel{Description:}
\acs{CMS} can request to change the operation state of a pedelec located at a slot of a \acs{PS}. The \acs{PS} can accept or reject the process the request. When rejected, the \acs{PS} must include a reason.

\marginlabel{URL:} \url{[BASE_PS_URI]/operation-state/pedelec}

\marginlabel{Method:} PUT

\marginlabel{Request:} 
\begin{table}[!h]
\vspace{-7mm}
\begin{tabularx}{\linewidth}{ | l | X | }
  \hline
  \textit{Field} & \textit{Description} \\
  \hline \hline
  slotPosition	& The position of the slot where the pedelec is located. \\
  pedelecManufacturerId		&  \\
  state 			& See Section \ref{types:OperationState} \\
  \hline
\end{tabularx}
\end{table}

\marginlabel{Response:} \textbf{202 Accepted}

\marginlabel{Errors:} \textbf{406 Not Acceptable}

\textbf{5xx Server Error}

\subsection{Change Configuration}
\label{cms:change-conf}

\marginlabel{Description:}
\acs{CMS} can request a \acs{PS} to change configuration parameters. This request contains a list of key-value pairs, where "key" is the name of the configuration setting to change and "value" contains the new setting for the configuration setting.

\marginlabel{URL:} \url{[BASE_PS_URI]/config}

\marginlabel{Method:} PUT

\newpage
\marginlabel{Request:} 
\begin{table}[!h]
\vspace{-7mm}
\begin{tabularx}{\linewidth}{ | l | l | X | }
  \hline
  \textit{Field} & \textit{Value} &\textit{Description} \\
  \hline \hline
  cmsURI 			& string 		& New value for the CMS Webservice URI \\
  heartBeatInterval 			& integer 		& In seconds \\
  openSlotTimeout 			& integer 		& In seconds. How long should \acs{PS} wait after unlocking a slot before locking it again. \\
  rebootRetries 				& integer 		& How many times should \acs{PS} try to reboot before giving up.\\
  chargingStatusInformInterval 	& integer 		& In seconds \\
  \hline
\end{tabularx}
\end{table}

\marginlabel{Response:} 
\textbf{200 OK} (If all the parameter changes are accepted and done)

\marginlabel{Errors:} \textbf{400 Bad Request}

\begin{tabularx}{\linewidth}{ | l | X | X | }
  \hline
  \textit{Field} & \textit{Value} & \textit{Description} \\
  \hline \hline
  failed			& JSON array & List of parameters that \acs{PS} failed to set a new value for.\\
  reason 		& NotAcceptable or NotFound & \\
  \hline
\end{tabularx}

\textit{NotAcceptable:} If the request for some keys could not be processed. It returns a JSON array of keys that are rejected (in this case other parameters are set)

\textit{NotFound:} If some of the keys are not found/supported. It returns a JSON array of keys that are not found as configuration parameters (in this case other parameters are set)


\textbf{5xx Server Error}

\subsection{Get Configuration}

\marginlabel{Description:}
\acs{CMS} can retrieve the values of configuration settings. This operation requires no parameters, and \acs{PS} returns all values.

\marginlabel{URL:} \url{[BASE_PS_URI]/config}

\marginlabel{Method:} GET

\marginlabel{Request:} -

\marginlabel{Response:} \textbf{200 OK}

JSON object with return values for elements defined in Section \ref{cms:change-conf}/Request

\marginlabel{Errors:} \textbf{5xx Server Error}

\subsection{Get Charging Status}

\marginlabel{Description:}
Even though a \acs{PS} informs the \acs{CMS} about the charging status of pedelecs regularly, it is desirable to get the latest information in various cases.

\marginlabel{URL:} \url{[BASE_PS_URI]/charging-status}

\marginlabel{Method:} GET

\marginlabel{Request:} -

\marginlabel{Response:} \textbf{200 OK}

JSON object with return values for elements defined in Section \ref{ps:charging-status}/Request

\marginlabel{Errors:} \textbf{5xx Server Error}

\subsection{Remote Authorize}

\marginlabel{Description:}
When using the mobile app for renting a pedelec the user does not require a card to authenticate against the \acs{PS}, but uses the app to authenticate directly against the \acs{CMS}. In this case, \acs{CMS} sends a Remote Authorize message to the \acs{PS} to unlock the slot(s) for the user to take the pedelec. 

After a timeout period \acs{PS} checks the existence of a pedelec at the slot(s) and sends a Start Transaction message to \acs{CMS}, namely the rental process proceeds usual.


\marginlabel{URL:} \url{[BASE_PS_URI]/authorize/remote}

\marginlabel{Method:} PUT

\marginlabel{Request:}
\begin{table}[!h]
\vspace{-7mm}
\begin{tabularx}{\linewidth}{ | l | X | }
  \hline
  \textit{Field} & \textit{Description} \\
  \hline \hline
  slotPosition (optional) 		&  \\
  userId		&  \\
  \hline
\end{tabularx}
\end{table}

\marginlabel{Response:} \textbf{202 Accepted}

\marginlabel{Errors:} \textbf{406 Not Acceptable}

\textbf{5xx Server Error}

\subsection{Cancel Authorize}

\marginlabel{Description:}
When using the mobile app for renting a pedelec the user can wish to cancel the rental process after Remote Authorize is initiated. In this case, \acs{CMS} sends a Cancel Authorize message to the \acs{PS} lock the slot(s) again.

\marginlabel{URL:} \url{[BASE_PS_URI]/authorize/cancel/<slotPosition>}

\marginlabel{Method:} PUT

\marginlabel{Request:} -

\marginlabel{Response:} \textbf{202 Accepted}

\marginlabel{Errors:} \textbf{406 Not Acceptable}

\textbf{5xx Server Error}

\subsection{Reboot}

\marginlabel{Description:}
\acs{CMS} can request a \acs{PS} to reboot. The \acs{PS} must respond with "Accepted" or "Rejected". When accepted, the \acs{PS} reboots after gracefully terminating running software. When rejected, the \acs{PS} must include a reason.

\marginlabel{URL:} \url{[BASE_PS_URI]/reboot}

\marginlabel{Method:} PUT

\marginlabel{Request:} -

\marginlabel{Response:} \textbf{202 Accepted}

\marginlabel{Errors:} \textbf{406 Not Acceptable}

\textbf{5xx Server Error}

\subsection{Unlock Slot}

\marginlabel{Description:}
In cases of maintenance or technical problems \acs{CMS} can request a \acs{PS} to unlock a slot or all slots in order to access a pedelec. 

\marginlabel{URL:} \url{[BASE_PS_URI]/unlock/<slotPosition>} \\
slotPosition is optional. When absent, the \acs{PS} unlocks all slots.

\marginlabel{Method:} PUT

\marginlabel{Request:} -

\marginlabel{Response:} \textbf{202 Accepted}

\marginlabel{Errors:} \textbf{406 Not Acceptable}

\textbf{5xx Server Error}
\section{Types}

%\subsection{Standard API errors}
%\label{types:errors}
%
%\begin{tabularx}{\linewidth}{ | l | X | }
%  \hline
%  \textit{Code} & \textit{Description} \\
%  \hline \hline
%  400 		& Bad Request \\
%  401 		& Unauthorized \\
%  \hline
%\end{tabularx}

\subsection{Error Message Template}
\label{types:error-msg-template}

In case an error occurs, regardless of the HTTP code that is returned the implementation \textit{always} responses with a JSON object containing following fields: 

\begin{tabularx}{\linewidth}{ | l | X | }
  \hline
  \rowcolor{table-head}
  Field & Description \\
  \hline
  timestamp & Unix timestamp (seconds since epoch)  \\
  code 		& Internal, application-specific error code \\
  message 	& Additional explanation \\
  \hline
\end{tabularx}


\subsection{Operation State}
\label{types:OperationState}

\begin{tabularx}{\linewidth}{ | l | X | }
  \hline
  \rowcolor{table-head}
  Value & Description \\
  \hline
  Operative 		& When the item is functional and working and ready to serve \\
  Inoperative 	& When the item is faulted and cannot be used \\
  \hline
\end{tabularx}

\subsection{Firmware Update State}
\label{types:FirmwareState}

\begin{tabularx}{\linewidth}{ | l | X | }
  \hline
  \rowcolor{table-head}
  Value & Description \\
  \hline
  DownloadFailed 		& \acs{PS} failed to load firmware \\
  InstallationFailed 	& Installation of firmware failed \\
  Installed				& Firmware is successfully installed in \acs{PS} \\
  \hline
\end{tabularx}

\subsection{Logs Update State}
\label{types:LogUpdateState}

\begin{tabularx}{\linewidth}{ | l | X | }
  \hline
  \rowcolor{table-head}
  Value & Description \\
  \hline
  Uploaded 		&  \\
  UploadFailed 	&  \\
  \hline
\end{tabularx}

\subsection{Configuration Error Reason}
\label{types:ConfigErrorReason}

\begin{tabularx}{\linewidth}{ | l | X | }
  \hline
  \rowcolor{table-head}
  Value & Description \\
  \hline
  NotAcceptable 		& If the request for some keys could not be processed. The server returns a JSON array of keys that are rejected (in this case other parameters are set) \\
  NotFound 	& If some of the keys are not found/supported. The server returns a JSON array of keys that are not found as configuration parameters (in this case other parameters are set) \\
  \hline
\end{tabularx}

%\subsection{Configuration Key-Value}
%\label{types:ConfigurationKeyValue}
%
%\begin{tabularx}{\linewidth}{ l l X }
%  \hline
%  \textit{Key} & \textit{Value} &\textit{Description} \\
%  \hline \hline
%  CentralSystemURL 	& string 	& New value for the URL \\
%\end{tabularx}

\end{document}


