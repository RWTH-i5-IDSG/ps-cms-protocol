\section{Operations Initiated by Pedelec Station}

\subsection{Boot Notification}

\marginlabel{Description:}
After start-up of a \acs{PS}, the \acs{PS} sends a notification to the \acs{CMS} with information about its configuration (e.g., manufacturer id, connected station slots and pedelecs). \acs{CMS} will accept only registered stations. 

After each reboot, the Boot Notification is sent.

The \acs{CMS} sends a response with the acceptable status, including current time and heartbeat interval if accepted.

The \acs{PS} repeats the Boot Notification (in an appropriated interval) until the \acs{CMS} accepts the \acs{PS}. The \acs{PS} requests nothing else, until \acs{CMS} accepts it.


\marginlabel{URL:} \url{[BASE_CMS_URI]/boot}

\marginlabel{Method:} POST

\marginlabel{Request:}
\begin{table}[!h]
\vspace{-7mm}
\begin{tabularx}{\linewidth}{ | l | l | X | }
  \hline
  \rowcolor{table-head}
  Field & Type & Description \\
  \hline
  stationManufacturerId & String 		& This value identifies the \acs{PS} by its hardware serial \\
  firmwareVersion			& String & Firmware version of \acs{PS}\\
  slots/slotManufacturerId 	& String & This value identifies the slot by its hardware serial \\
  slots/slotPosition			& Integer & The sequence position of the connected slot \\
  slots/pedelecManufacturerId & String (optional) & This value identifies the locked pedelec by its hardware serial related to the slot \\
  stationURL				& String & URL to communicate with \acs{PS} initiated by \acs{CMS}\\
    \hline
\end{tabularx}
\end{table}

\marginlabel{Response:}
\textbf{200 OK}

\begin{tabularx}{\linewidth}{ | l | l | X | }
  \hline
  \rowcolor{table-head}
  Field & Type & Description \\
  \hline
  timestamp & Long & Unix timestamp (seconds since epoch) \\
  heartbeatInterval & Integer & In seconds \\
  cardKeys/name 		& String & Name of the key pair for identification \\
  cardKeys/readKey 	& String & Encryption key to read out user's cardId \\
  cardKeys/writeKey 	& String & Encryption key to write user's cardId \\
  \hline
\end{tabularx}

\marginlabel{Errors:} 
\textbf{406 Not Acceptable} (If the station is not registered)

\textbf{5xx Server Error}

\subsection{Station Status Notification}

\marginlabel{Description:}
A \acs{PS} sends a notification to the \acs{CMS} to inform the \acs{CMS} about its status or error condition within the \acs{PS} including the connected station slots. A \acs{PS} shall send a Station Status Notification when it becomes unavailable as a result of an error condition or other external events.

\marginlabel{URL:} \url{[BASE_CMS_URI]/status/station}

\marginlabel{Method:} POST

\marginlabel{Request:}
\begin{table}[!h]
\vspace{-7mm}
\begin{tabularx}{\linewidth}{ | l  | l | X | }
  \hline
  \rowcolor{table-head}
  Field & Type & Description \\
  \hline
  stationManufacturerId & String 		& This value identifies the \acs{PS} by its hardware serial\\
  stationErrorCode & String & Required when stationState is INOPERATIVE \\
  stationErrorInfo & String & Required when stationState is INOPERATIVE \\
  stationState & String & See Section \ref{types:OperationState} \\
  timestamp & Long & Unix timestamp (seconds since epoch) \\
  slots/slotManufacturerId & String 	& This value identifies the slot by its hardware serial \\
  slots/slotErrorCode & String & Required when slotState is INOPERATIVE \\
  slots/slotErrorInfo & String & Required when slotState is INOPERATIVE \\
  slots/slotState & String & See Section \ref{types:OperationState} \\
  \hline
\end{tabularx}
\end{table}

\marginlabel{Response:} \textbf{200 OK}

\marginlabel{Errors:} \textbf{5xx Server Error}

\subsection{Pedelec Status Notification}

\marginlabel{Description:}
A \acs{PS} sends a notification to the \acs{CMS} to inform the \acs{CMS} about the status or error condition of connected pedelecs. A \acs{PS} shall send an Pedelec Status Notification when a pedelec becomes unavailable as a result of an error condition or other external events.

\marginlabel{URL:} \url{[BASE_CMS_URI]/status/pedelec}

\marginlabel{Method:} POST

\marginlabel{Request:}
\begin{table}[!h]
\vspace{-7mm}
\begin{tabularx}{\linewidth}{ | l | l | X | }
  \hline
  \rowcolor{table-head}
  Field & Type & Description \\
  \hline
  pedelecManufacturerId & String 		& This value identifies the Pedelec by its hardware serial\\
  pedelecErrorCode & String & Required when pedelecState is INOPERATIVE \\
  pedelecErrorInfo & String & Required when pedelecState is INOPERATIVE \\
  pedelecState & String & See Section \ref{types:OperationState} \\
  timestamp & Long & Unix timestamp (seconds since epoch) \\
  \hline
\end{tabularx}
\end{table}

\marginlabel{Response:} \textbf{200 OK}

\marginlabel{Errors:} \textbf{5xx Server Error}


\subsection{Charging Status Notification}
\label{ps:charging-status}

\marginlabel{Description:}
The \acs{PS} informs the \acs{CMS} at regular intervals about the charging status (in time intervals or when fully charged) of all connected pedelecs of the \acs{PS}.

The message contains the current timestamp, the meter value (Wh), the charging state (e.g., charging, completed), the and PedelecID.

\marginlabel{URL:} \url{[BASE_CMS_URI]/status/charging}

\marginlabel{Method:} POST

\marginlabel{Request:}
\begin{table}[!h]
\vspace{-7mm}
\begin{tabularx}{\linewidth}{ | l | l | X | }
  \hline
  \rowcolor{table-head}
  Field & Type & Description \\
  \hline
  	pedelecManufacturerId & String			& \\
  	timestamp & Long			& Unix timestamp (seconds since epoch) \\
  	chargingState & String		& See Section \ref{types:ChargingState} \\	
  	meterValue & Double & \\
  	battery/soc & Double & percentage points \\
  	battery/temperature & Double &  \\
  	battery/cycleCount & Integer &  \\
  	battery/voltage & Double &  \\
  	battery/current & Double &  \\
    \hline
\end{tabularx}
\end{table}

\marginlabel{Response:} \textbf{200 OK}

\marginlabel{Errors:} \textbf{5xx Server Error}

\subsection{Firmware Status Notification}

\marginlabel{Description:}
A \acs{PS} notifies \acs{CMS} about the success/failure of the firmware update. 

\marginlabel{URL:} \url{[BASE_CMS_URI]/status/firmware}

\marginlabel{Method:} POST

\marginlabel{Request:} 
\begin{tabularx}{\linewidth}{ | l | l | X | }
  \hline
  \rowcolor{table-head}
  Field & Type & Description \\
  \hline
  	status & String			& Progress status of the firmware update; see Section \ref{types:FirmwareState} \\	
  \hline
\end{tabularx}

\marginlabel{Response:} \textbf{200 OK}

\marginlabel{Errors:} \textbf{5xx Server Error}

\subsection{Logs Status Notification}

\marginlabel{Description:}
The \acs{PS} informs the \acs{CMS} about the status of requested uploading of logs.

\marginlabel{URL:} \url{[BASE_CMS_URI]/status/logs}

\marginlabel{Method:} POST

\marginlabel{Request:} 
\begin{tabularx}{\linewidth}{ | l | l | X | }
  \hline
  \rowcolor{table-head}
  Field & Type & Description \\
  \hline
  	status & String			& Upload status of the logs; see Section \ref{types:LogUpdateState} \\	
  \hline
\end{tabularx}

\marginlabel{Response:} \textbf{200 OK}

\marginlabel{Errors:} \textbf{5xx Server Error}

\subsection{Authorize}

\marginlabel{Description:}
Before a user can choose and unlock a pedelec with his CustomerCard (e.g., Bluecard), the \acs{PS} needs to be able to authorize the operation. Only after authorization the \acs{PS} will be able to unlock the pedelec. For this purpose the \acs{PS} needs user's Card-ID and PIN for authorization.

The response shall indicate, whether or not the Card-ID and PIN combination is accepted by the \acs{CMS}.

\marginlabel{URL:} \url{[BASE_CMS_URI]/authorize}

\marginlabel{Method:} POST

\marginlabel{Request:}
\begin{table}[!h]
\vspace{-7mm}
\begin{tabularx}{\linewidth}{ | l | l |  X | }
  \hline
  \rowcolor{table-head}
  Field & Type & Description \\
  \hline
  	cardId & String 		& Card specific number\\
  	cardPin & String			& User's secret PIN \\
    \hline
\end{tabularx}
\end{table}

\marginlabel{Response:}
\textbf{202 Accepted} (If credentials are accepted)

\begin{table}[!h]
\begin{tabularx}{\linewidth}{ | l | l | X | }
  \hline
  \rowcolor{table-head}
  Field & Type & Description \\
  \hline
  	cardId & String 		& Card specific number\\
  	accountState & Account State & See Section \ref{types:AccountState} \\
%  	trialsRemaining & Count of remaining authentication trials, before the account gets disabled.\\
    \hline
\end{tabularx}
\end{table}
\marginlabel{Errors:}
\textbf{403 Forbidden - no trials remaining and account gets disabled}

\textbf{403 Forbidden - card account is unknown}

\textbf{403 Forbidden - card account is disabled}

\textbf{403 Forbidden - wrong pin}

\textbf{406 Not Acceptable}

\textbf{5xx Server Error}

\subsection{Activate Card}

\marginlabel{Description:}
Before a user can use his card (e.g., Bluecard) to rent a bike, he has to activate it on the \acs{PS} terminal. For this purpose, the \acs{PS} sends Activation-Key and PIN to the \acs{CMS}.

The response shall indicate, whether or not the Activation-Key and PIN are accepted by the \acs{CMS} and responses with the CardId.

\marginlabel{URL:} \url{[BASE_CMS_URI]/activate-card}

\marginlabel{Method:} POST

\marginlabel{Request:}
\begin{table}[!h]
\vspace{-7mm}
\begin{tabularx}{\linewidth}{ | l | l | X | }
  \hline
  \rowcolor{table-head}
  Field & Type & Description \\
  \hline
  	activationKey & String & Key to start activation process to initial user's card\\
  	cardPin & String & PIN for customer's card\\
    \hline
\end{tabularx}
\end{table}

\marginlabel{Response:}
\textbf{202 Accepted} (If credentials are accepted)

\begin{table}[!h]
\begin{tabularx}{\linewidth}{ | l | l | X | }
  \hline
  \rowcolor{table-head}
  Field & Type & Description \\
  \hline
  	cardId 		& String & Card specific number\\
    \hline
\end{tabularx}
\end{table}

\marginlabel{Errors:} \textbf{406 Not Acceptable}

\textbf{5xx Server Error}

%\subsection{Set Card-Pin}
%
%\marginlabel{Description:}
%
%After a successful validation of the activation-key, the user can type in a pin for his card. At the end of the process, the \acs{PS} initializes the card with the userId. Is the copy procedure finished, the \acs{PS} sends cardId and cardPin to the \acs{CMS}.
%
%The response shall indicate, whether or not the card pin is successfully updated.
%
%\marginlabel{URL:} \url{[BASE_CMS_URI]/set-card-pin}
%
%\marginlabel{Method:} POST
%
%\marginlabel{Request:}
%\begin{table}[!h]
%\vspace{-7mm}
%\begin{tabularx}{\linewidth}{ | l | X | }
%  \hline
%  \rowcolor{table-head}
%  Field & Description \\
%  \hline
%  	cardId & Card Id to identifies user's card\\
%  	cardPin & Card Pin to set pin for user's card\\
%    \hline
%\end{tabularx}
%\end{table}
%
%\marginlabel{Response:}
%\textbf{200 OK} (If credentials are accepted)
%
%\marginlabel{Errors:} \textbf{406 Not Acceptable}
%
%\textbf{5xx Server Error}


\subsection{Start Transaction}

\marginlabel{Description:}
When the rental is authenticated, the station slot is unlocked and the user took the pedelec out of the slot, the \acs{PS} needs to inform the \acs{CMS} about this.

As response the \acs{CMS} sends an acknowledgment.

\marginlabel{URL:} \url{[BASE_CMS_URI]/transaction/start}

\marginlabel{Method:} POST

\marginlabel{Request:} 
\begin{table}[!h]
\vspace{-7mm}
\begin{tabularx}{\linewidth}{ | l | l | X | }
  \hline
  \rowcolor{table-head}
  Field & Type & Description \\
  \hline
  	cardId & String 		& \\
  	pedelecManufacturerId & String			& \\
  	stationManufacturerId & String			& \\
  	slotManufacturerId & String			& \\
  	timestamp & Long					& Unix timestamp (seconds since epoch) \\
    \hline
\end{tabularx}
\end{table}

\marginlabel{Response:} \textbf{200 OK}

\marginlabel{Errors:} \textbf{5xx Server Error}

\subsection{Stop Transaction}

\marginlabel{Description:}
After the \acs{PS} recognizes the return of a pedelec at a station slot, it needs to inform the \acs{CMS} about this.

As response the \acs{CMS} sends an acknowledgment.

\marginlabel{URL:} \url{[BASE_CMS_URI]/transaction/stop}

\marginlabel{Method:} POST

\marginlabel{Request:} 
\begin{table}[!h]
\vspace{-7mm}
\begin{tabularx}{\linewidth}{ | l | l | X | }
  \hline
  \rowcolor{table-head}
  Field & Type & Description \\
  \hline
  	pedelecManufacturerId			& String & \\
  	stationManufacturerId			& String & \\
  	slotManufacturerId			& String & \\
  	timestamp					& Long & Unix timestamp (seconds since epoch) \\
  	
    \hline
\end{tabularx}
\end{table}

\marginlabel{Response:} \textbf{200 OK}

\marginlabel{Errors:} \textbf{5xx Server Error}

\subsection{Get Available Pedelecs}

\marginlabel{Description:}
\acs{PS} can retrieve a list of available pedelecs ordered by a predefined priority\\ or a specific pedelec for a user's reservation when available for this moment.

\marginlabel{URL:} \url{[BASE_CMS_URI]/available-pedelecs?cardId=ab34-cd56}

\marginlabel{Method:} GET

\marginlabel{Request:}
\begin{table}[!h]
\vspace{-7mm}
\begin{tabularx}{\linewidth}{ | l | l | X | }
  \hline
  \rowcolor{table-head}
  Field & Type & Description \\
  \hline
  	cardId & String & User's card ID\\
    \hline
\end{tabularx}
\end{table}


\marginlabel{Response:} \textbf{200 OK}

\begin{tabularx}{\linewidth}{ | l | X | }
  \hline
  \rowcolor{table-head}
  Type & Description \\
  \hline
  	String[] & List of manufacturer Ids for available pedelecs \\	
  \hline
\end{tabularx}

\marginlabel{Errors:} \textbf{5xx Server Error}

\subsection{Heartbeat}

\marginlabel{Description:}
To let the \acs{CMS} know that a station is still connected, a \acs{PS} sends heartbeats regularly in configurable time intervals.

The \acs{CMS} sends a response with the current time of the \acs{CMS}, which could be used to synchronize the time of the \acs{PS} with the time of the \acs{CMS}.

\marginlabel{URL:} \url{[BASE_CMS_URI]/heartbeat}

\marginlabel{Method:} GET

\marginlabel{Request:} -

\marginlabel{Response:} \textbf{200 OK}

\begin{tabularx}{\linewidth}{ | l | l | X | }
  \hline
  \rowcolor{table-head}
  Field & Type & Description \\
  \hline
  	timestamp & Long			& Unix timestamp (seconds since epoch) \\	
  \hline
\end{tabularx}


\marginlabel{Errors:} \textbf{5xx Server Error}
