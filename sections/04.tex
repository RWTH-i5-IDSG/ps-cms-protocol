\section{Types}

%\subsection{Standard API errors}
%\label{types:errors}
%
%\begin{tabularx}{\linewidth}{ | l | X | }
%  \hline
%  \textit{Code} & \textit{Description} \\
%  \hline \hline
%  400 		& Bad Request \\
%  401 		& Unauthorized \\
%  \hline
%\end{tabularx}

\subsection{Error Message Template}
\label{types:error-msg-template}

In case an error occurs, regardless of the HTTP code that is returned the implementation \textit{always} responses with a JSON object containing following fields: 

\begin{tabularx}{\linewidth}{ | l | X | }
  \hline
  \textit{Field} & \textit{Description} \\
  \hline \hline
  timestamp & Unix timestamp (seconds since epoch)  \\
  code 		& Internal, application-specific error code \\
  message 	& Additional explanation \\
  \hline
\end{tabularx}


\subsection{Operation State}
\label{types:OperationState}

\begin{tabularx}{\linewidth}{ | l | X | }
  \hline
  \textit{Value} & \textit{Description} \\
  \hline \hline
  Operative 		& When the item is functional and working and ready to serve \\
  Inoperative 	& When the item is faulted and cannot be used \\
  \hline
\end{tabularx}

\subsection{Firmware State}
\label{types:FirmwareState}

\begin{tabularx}{\linewidth}{ | l | X | }
  \hline
  \textit{Value} & \textit{Description} \\
  \hline \hline
  DownloadFailed 		& PS failed to load firmware \\
  InstallationFailed 	& Installation of firmware failed \\
  Installed				& Firmware is successfully installed in PS \\
  \hline
\end{tabularx}

%\subsection{Configuration Key-Value}
%\label{types:ConfigurationKeyValue}
%
%\begin{tabularx}{\linewidth}{ l l X }
%  \hline
%  \textit{Key} & \textit{Value} &\textit{Description} \\
%  \hline \hline
%  CentralSystemURL 	& string 	& New value for the URL \\
%\end{tabularx}