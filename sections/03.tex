\section{Operations Initiated by Central Management System}

\subsection{Change Station Operation State}

\marginlabel{Description:}
\acs{CMS} can request to change the operation state of a \acs{PS} or its slots. The \acs{PS} can accept or reject the process the request. When rejected, the \acs{PS} must include a reason.

\marginlabel{URL:} \url{[BASE_PS_URI]/state/station}

\marginlabel{Method:} POST

\marginlabel{Request:} 
\begin{table}[!h]
\vspace{-7mm}
\begin{tabularx}{\linewidth}{ | l | X | }
  \hline
  \textit{Field} & \textit{Description} \\
  \hline \hline
  slotPosition (optional) 		& When present, the state of the slot with the given position will be changed. When absent, the state of whole \acs{PS} will be changed. \\
  state 					& See Section \ref{types:OperationState} \\
    \hline
\end{tabularx}
\end{table}

\marginlabel{Response:} \textbf{202 Accepted}

\marginlabel{Errors:} \textbf{406 Not Acceptable}

\textbf{5xx Server Error}

\subsection{Change Pedelec Operation State}

\marginlabel{Description:}
\acs{CMS} can request to change the operation state of a pedelec located at a slot of a \acs{PS}. The \acs{PS} can accept or reject the process the request. When rejected, the \acs{PS} must include a reason.

\marginlabel{URL:} \url{[BASE_PS_URI]/state/pedelec}

\marginlabel{Method:} POST

\marginlabel{Request:} 
\begin{table}[!h]
\vspace{-7mm}
\begin{tabularx}{\linewidth}{ | l | X | }
  \hline
  \textit{Field} & \textit{Description} \\
  \hline \hline
  slotPosition	& The position of the slot where the pedelec is located. \\
  pedelecManufacturerId		&  \\
  pedelecState 			& See Section \ref{types:OperationState} \\
  \hline
\end{tabularx}
\end{table}

\marginlabel{Response:} \textbf{202 Accepted}

\marginlabel{Errors:} \textbf{406 Not Acceptable}

\textbf{5xx Server Error}

\subsection{Change Pedelec Configuration}
\label{cms:change-pedelec-conf}

\marginlabel{Description:}
\acs{CMS} can request a \acs{PS} to change specific Pedelec configuration parameters. This request contains a list of key-value pairs, where "key" is the name of the configuration setting to change and "value" contains the new setting for the configuration setting.

\marginlabel{URL:} \url{[BASE_PS_URI]/pedelec/<pedelecManufacturerId>/config}

\marginlabel{Method:} POST

\newpage
\marginlabel{Request:} 
\begin{table}[!h]
\vspace{-7mm}
\begin{tabularx}{\linewidth}{ | l | l | X | }
  \hline
  \textit{Field} & \textit{Value} &\textit{Description} \\
  \hline \hline

  \hline
\end{tabularx}
\end{table}

\marginlabel{Response:} 
\textbf{202 Accepted} (If all the parameter changes are accepted and done)

\marginlabel{Errors:} \textbf{400 Bad Request}

\begin{tabularx}{\linewidth}{ | l | X | X | }
  \hline
  \textit{Field} & \textit{Value} & \textit{Description} \\
  \hline \hline
  failed			& JSON array & List of parameters that \acs{PS} failed to set a new value for.\\
  reason 		& NotAcceptable or NotFound & \\
  \hline
\end{tabularx}

\textit{NotAcceptable:} If the request for some keys could not be processed. It returns a JSON array of keys that are rejected (in this case other parameters are set)

\textit{NotFound:} If some of the keys are not found/supported. It returns a JSON array of keys that are not found as configuration parameters (in this case other parameters are set)


\textbf{5xx Server Error}

\subsection{Get Pedelec Configuration}

\marginlabel{Description:}
\acs{CMS} can retrieve the values of configuration settings. This operation requires no parameters, and \acs{PS} returns all values.

\marginlabel{URL:} \url{[BASE_PS_URI]/pedelec/<pedelecManufacturerId>/config}

\marginlabel{Method:} GET

\marginlabel{Request:} -

\marginlabel{Response:} \textbf{200 OK}

JSON object with return values for Pedelec-Configuration

\marginlabel{Errors:} \textbf{5xx Server Error}

\subsection{Change Station Configuration}
\label{cms:change-conf}

\marginlabel{Description:}
\acs{CMS} can request a \acs{PS} to change configuration parameters. This request contains a list of key-value pairs, where "key" is the name of the configuration setting to change and "value" contains the new setting for the configuration setting.

\marginlabel{URL:} \url{[BASE_PS_URI]/station/config}

\marginlabel{Method:} POST

\newpage
\marginlabel{Request:} 
\begin{table}[!h]
\vspace{-7mm}
\begin{tabularx}{\linewidth}{ | l | l | X | }
  \hline
  \textit{Field} & \textit{Value} &\textit{Description} \\
  \hline \hline
  cmsURI 			& string 		& New value for the \acs{CMS} Webservice URI \\
  heartBeatInterval 			& integer 		& In seconds \\
  openSlotTimeout 			& integer 		& In seconds. How long should \acs{PS} wait after unlocking a slot before locking it again. \\
  rebootRetries 				& integer 		& How many times should \acs{PS} try to reboot before giving up.\\
  chargingStatusInformInterval 	& integer 		& In seconds \\
  \hline
\end{tabularx}
\end{table}

\marginlabel{Response:} 
\textbf{202 Accepted} (If all the parameter changes are accepted and done)

\marginlabel{Errors:} \textbf{400 Bad Request}

\begin{tabularx}{\linewidth}{ | l | X | X | }
  \hline
  \textit{Field} & \textit{Value} & \textit{Description} \\
  \hline \hline
  failed			& JSON array & List of parameters that \acs{PS} failed to set a new value for.\\
  reason 		& NotAcceptable or NotFound & \\
  \hline
\end{tabularx}

\textit{NotAcceptable:} If the request for some keys could not be processed. It returns a JSON array of keys that are rejected (in this case other parameters are set)

\textit{NotFound:} If some of the keys are not found/supported. It returns a JSON array of keys that are not found as configuration parameters (in this case other parameters are set)


\textbf{5xx Server Error}

\subsection{Get Station Configuration}

\marginlabel{Description:}
\acs{CMS} can retrieve the values of configuration settings. This operation requires no parameters, and \acs{PS} returns all values.

\marginlabel{URL:} \url{[BASE_PS_URI]/station/config}

\marginlabel{Method:} GET

\marginlabel{Request:} -

\marginlabel{Response:} \textbf{200 OK}

JSON object with return values for elements defined in Section \ref{cms:change-conf}/Request

\marginlabel{Errors:} \textbf{5xx Server Error}

\subsection{Get Charging Status}

\marginlabel{Description:}
Even though a \acs{PS} informs the \acs{CMS} about the charging status of pedelecs regularly, it is desirable to get the latest information in various cases.

\marginlabel{URL:} \url{[BASE_PS_URI]/charging-status}

\marginlabel{Method:} GET

\marginlabel{Request:} -

\marginlabel{Response:} \textbf{200 OK}

JSON object with return values for elements defined in Section \ref{ps:charging-status}/Request

\marginlabel{Errors:} \textbf{5xx Server Error}

\subsection{Remote Authorize}

\marginlabel{Description:}
When using the mobile app for renting a pedelec the user does not require a card to authenticate against the \acs{PS}, but uses the app to authenticate directly against the \acs{CMS}. In this case, \acs{CMS} sends a Remote Authorize message to the \acs{PS} to unlock the slot(s) for the user to take the pedelec. 

After a timeout period \acs{PS} checks the existence of a pedelec at the slot(s) and sends a Start Transaction message to \acs{CMS}, namely the rental process proceeds usual.


\marginlabel{URL:} \url{[BASE_PS_URI]/authorize/remote}

\marginlabel{Method:} POST

\marginlabel{Request:}
\begin{table}[!h]
\vspace{-7mm}
\begin{tabularx}{\linewidth}{ | l | X | }
  \hline
  \textit{Field} & \textit{Description} \\
  \hline \hline
  slotPosition 		&  \\
  userId		&  \\
  \hline
\end{tabularx}
\end{table}

\marginlabel{Response:} \textbf{202 Accepted}

\marginlabel{Errors:} \textbf{406 Not Acceptable}

\textbf{5xx Server Error}

\subsection{Cancel Authorize}

\marginlabel{Description:}
When using the mobile app for renting a pedelec the user can wish to cancel the rental process after Remote Authorize is initiated. In this case, \acs{CMS} sends a Cancel Authorize message to \acs{PS} to lock the slot(s) again.

\marginlabel{URL:} \url{[BASE_PS_URI]/authorize/cancel/<slotPosition>}

\marginlabel{Method:} POST

\marginlabel{Request:} -

\marginlabel{Response:} \textbf{202 Accepted}

\marginlabel{Errors:} \textbf{406 Not Acceptable}

\textbf{5xx Server Error}

\subsection{Reboot}

\marginlabel{Description:}
\acs{CMS} can request a \acs{PS} to reboot. When accepted, the \acs{PS} reboots after gracefully terminating running software. When rejected, the \acs{PS} must include a reason.

\marginlabel{URL:} \url{[BASE_PS_URI]/reboot}

\marginlabel{Method:} POST

\marginlabel{Request:} -

\marginlabel{Response:} \textbf{202 Accepted}

\marginlabel{Errors:} \textbf{406 Not Acceptable}

\textbf{5xx Server Error}

\subsection{Unlock Slot}

\marginlabel{Description:}
In cases of maintenance or technical problems \acs{CMS} can request a \acs{PS} to unlock a slot or all slots in order to access a pedelec. 

\marginlabel{URL:} \url{[BASE_PS_URI]/unlock/<slotPosition>} \\
slotPosition is optional. When absent, the \acs{PS} unlocks all slots.

\marginlabel{Method:} POST

\marginlabel{Request:} -

\marginlabel{Response:} \textbf{202 Accepted}

\marginlabel{Errors:} \textbf{406 Not Acceptable}

\textbf{5xx Server Error}

\subsection{Update Firmware}

\marginlabel{Description:}
The \acs{CMS} can send the \acs{PS} a firmware update command. With a firmware update request, the \acs{CMS} informs the \acs{PS} about a new firmware, including the location of the firmware and the date \& time when the update shall be executed.

The \acs{PS} should start as soon as possible the firmware after retrieving the retrieve-update.

\marginlabel{URL:} \url{[BASE_PS_URI]/update-firmware}

\marginlabel{Method:} POST

\marginlabel{Request:}
\begin{table}[!h]
\vspace{-7mm}
\begin{tabularx}{\linewidth}{ | l | X | }
  \hline
  \textit{Field} & \textit{Description} \\
  \hline \hline
  firmwareUpdateUrl 		& Location of the firmware update \\
  executionDateTime	&  Date and time when the \acs{PS} should execute update \\
  \hline
\end{tabularx}
\end{table}

\marginlabel{Response:} \textbf{200 OK}

\marginlabel{Errors:} \textbf{5xx Server Error}