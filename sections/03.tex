\section{Operations Initiated by Central Management System}

\subsection{Change Station Operation State}

\acs{CMS} can request to change the operation state of a \acs{PS} or its slots. The \acs{PS} must respond with "Accepted" or "Rejected". When rejected, the \acs{PS} must include a reason.

\seealso{Section \ref{types:OperationState}} See Section \ref{types:OperationState} for allowed operation states.

\subsection{Change Pedelec Operation State}

\acs{CMS} can request to change the operation state of a pedelec located at a slot of a \acs{PS}. The \acs{PS} must respond with "Accepted" or "Rejected". When rejected, the \acs{PS} must include a reason.

\seealso{Section \ref{types:OperationState}} See Section \ref{types:OperationState} for allowed operation states.

\subsection{Change Configuration}

\acs{CMS} can request a \acs{PS} to change configuration parameters. This request contains a list of key-value pairs, where "key" is the name of the configuration setting to change and "value" contains the new setting for the configuration setting. The \acs{PS} must respond with one of the following:
\begin{itemize}
	\item "Accepted" if all the parameter changes are accepted and done.
	\item "Rejected" with a list of keys that are rejected (in this case other parameters are set)
	\item "Not Found" with a list of keys that are not found as configuration parameters (in this case other parameters are set)
\end{itemize}

\seealso{Section \ref{types:ConfigurationKeyValue}} See Section \ref{types:ConfigurationKeyValue} for allowed configuration parameters.

\subsection{Get Configuration}

\acs{CMS} can retrieve the values of configuration settings. This operation requires no parameters, and \acs{PS} returns all values.

\seealso{Section \ref{types:ConfigurationKeyValue}} See Section \ref{types:ConfigurationKeyValue} for returned configuration parameters.

\subsection{Remote Authorize}

When using the mobile app for renting a pedelec the user does not require a card to authenticate against the \acs{PS}, but uses the app to authenticate directly against the \acs{CMS}. In this case, \acs{CMS} sends a Remote Authorize message to the \acs{PS} to unlock the slot(s) for the user to take the pedelec. 

After a timeout period \acs{PS} checks the existence of a pedelec at the slot(s) and sends a Start Transaction message to \acs{CMS}, namely the rental process proceeds usual.

\subsection{Cancel Authorize}

When using the mobile app for renting a pedelec the user can wish to cancel the rental process after Remote Authorize is initiated. In this case, \acs{CMS} sends a Cancel Authorize message to the \acs{PS} lock the slot(s) again.

\subsection{Reboot}

\acs{CMS} can request a \acs{PS} to reboot. The \acs{PS} must respond with "Accepted" or "Rejected". When accepted, the \acs{PS} reboots after gracefully terminating running software. When rejected, the \acs{PS} must include a reason.

\subsection{Unlock Slot}

In cases of maintenance or technical problems \acs{CMS} can request a \acs{PS} to unlock a slot in order to access a pedelec.

\subsection{Get Charge State}

Even though a \acs{PS} informs the \acs{CMS} about the charging state of pedelecs regularly, it is desirable to get the latest information in various cases.